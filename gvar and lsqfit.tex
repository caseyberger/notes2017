\documentclass[english,course,headertitle,theoremsection,nocleardoublepage,twoside,onecolumn]{Notes}
\usepackage{hyperref,mathtools,amssymb}
\usepackage{listings}
\usepackage{color}

\definecolor{dkgreen}{rgb}{0,0.6,0}
\definecolor{gray}{rgb}{0.5,0.5,0.5}
\definecolor{mauve}{rgb}{0.58,0,0.82}

\lstset{frame=tb,
  language=Java,
  aboveskip=3mm,
  belowskip=3mm,
  showstringspaces=false,
  columns=flexible,
  basicstyle={\small\ttfamily},
  numbers=none,
  numberstyle=\tiny\color{gray},
  keywordstyle=\color{blue},
  commentstyle=\color{dkgreen},
  stringstyle=\color{mauve},
  breaklines=true,
  breakatwhitespace=true,
  tabsize=3
}

%http://blog.poormansmath.net/latex-class-for-lecture-notes/

\title{gvar and lsqfit}
%\subject{Computer Science}
\author{Casey E. Berger}
\date{\day}{\month}{\year}


\newcommand{\etal}{{\it et al.}}
\newcommand{\beq}{\begin{equation}}
\newcommand{\eeq}{\end{equation}}
\newcommand{\bea}{\begin{eqnarray}}
\newcommand{\eea}{\end{eqnarray}}
\newcommand{\inlinecode}{\texttt}

\def\CP{{\mathcal P}}
\def\CC{{\mathcal C}}
\def\CW{{\mathcal W}}
\def\CO{{\mathcal O}}
\def\CZ{{\mathcal Z}}
\def\CD{{\mathcal D}}
\def\Del{{\bigtriangledown}}

\begin{document}
\section{gvar}
\subsection{Dependencies}
No dependencies, but numpy is needed for array manipulation.
\subsection{Basics}
\subsubsection{Creating gvar variables}
To create a single gvar:
\lstset{language=python}
\begin{lstlisting}
import gvar as gv
#create a variable with mean 1.0 and standard deviation 0.2
x = gv.gvar(1.0,0.2)
x_mean = x.mean
x_std = x.sdev
y = gv.gvar(2.0,0.5)
y_mean = y.mean
y_std = y.sdev
print x
print y
print x + y
print x - y
print x*y
print x/y
print y*y
\end{lstlisting}

Variables created like this can be added, subtracted, multiplied, or divided, and the error will be automatically propagated.

You can also create a gvar by feeding in a string formatted in one of two ways:

\begin{lstlisting}
import gvar as gv
import numpy as np
#create a single gvar by feeding in a string formatted in one of two ways
s = "1.00(20)"
gv_s = gv.gvar(s)
print gv_s
s2 = "1.00 +- 0.2"
gv_s2 = gv.gvar(s2)
print gv_s2
\end{lstlisting}

Arrays or dictionaries of numbers can also be averaged together to get a gvar:

\begin{lstlisting}
import gvar as gv
import numpy as np
#create a gvar by averaging an array of floats
a = np.array([1.0,2.0,3.0,4.0,5.0])
gv_a = gv.dataset.avg_data(a)
print gv_a
\end{lstlisting}

The covariance matrix can also be extracted using gvar:
\begin{lstlisting}
import gvar as gv
import numpy as np

#extract a covariance matrix
b = np.array([gv.gvar(1.0,0.2),gv.gvar(1.5,0.3),gv.gvar(1.2,0.2),gv.gvar(2.0,0.4)])
cov_b = gv.evalcov(b)
print cov_b
\end{lstlisting}
In addition to using an array, a dictionary can be used to store the data.\\
\\
An array of gvar can be generated more quickly in a few different ways:
\begin{lstlisting}
import gvar as gv
import numpy as np

#create an array of gvar by feeding in the means and the covariance matrix
data = gv.gvar([1.0, 1.5,1.2,2.0], cov_b)
print data

#create an array of gvar by feeding in the means and the sdevs
data2 = gv.gvar([1.0, 1.5,1.2,2.0], [0.2,0.3,0.2,0.4])
print data2
\end{lstlisting}


You can also feed in a dictionary or an array.
\subsection{Frequent or Cryptic Error Messages}

\section{lsqfit}
\subsection{Dependencies}
This package requires numpy, gvar, and cython to run.
\subsection{Basics}
This fitter works for anything that can be fit using $\chi^{2}$ minimization.\\
\\
lsqfit's nonlinear fit function takes input y data in the form of an array of gvar objects, input x data in the form of an array of floats, priors in the form of an array of gvar objects, and a fit function, as well as other optional inputs and fits the function to the data.\\
\\
Here is an example of a simple exponential fit:
\begin{lstlisting}
import numpy as np
import gvar as gv
import lsqfit

def priors(): 
	p = dict()
	p['A0'] = [0.1,0.05]
	p['B0'] = [5.,0.5]
	p['C0'] = [3.5,0.5]
	prior = dict()
	for k in p.keys():
		prior[k] = gv.gvar(p[k][0], p[k][1])
	return prior

class fit_function():
	#def __init__(self):
	#	return None
	def A(self,n,p):
		return p['A%s' %n] 
	def B(self,n,p):
		return p['B%s' %n] 
	def C(self,n,p):
		return p['C%s' %n] 
	def func(self,t,p):
		r = 0
		for n in range(0,1):
			An = self.A(n,p)
			Bn = self.B(n,p)
			Cn = self.C(n,p)
			r += An + Bn*np.exp(-Cn*t)
		return r

x = np.arange(1,5) 
y = gv.gvar([0.26,0.099,0.10,0.09],[0.05,0.07,0.005,0.000007])
p = priors()
fitc = fit_function()
fit = lsqfit.nonlinear_fit(data=(x,y),prior=p,fcn=fitc.func)
print fit
\end{lstlisting}

\subsection{Frequent or Cryptic Error Messages}




\end{document}